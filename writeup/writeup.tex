\documentclass{article}
\title{Guessing game assessment}
\author{Izaak van Dongen}
\usepackage{amsmath}
\usepackage{amsfonts}

\usepackage[parfill]{parskip}

\usepackage{hyperref}

\usepackage[utf8]{inputenc}

\usepackage{listings}
\usepackage{color}

\definecolor{codegreen}{rgb}{0,0.6,0}
\definecolor{codegray}{rgb}{0.5,0.5,0.5}
\definecolor{codepurple}{rgb}{0.58,0,0.82}
\definecolor{backcolour}{rgb}{0.95,0.95,0.92}

\lstdefinestyle{mystyle}{
    backgroundcolor=\color{backcolour},   
    commentstyle=\color{codegreen},
    keywordstyle=\color{magenta},
    numberstyle=\tiny\color{codegray},
    stringstyle=\color{codepurple},
    basicstyle=\footnotesize,
    breaklines=true,                 
    captionpos=b,                    
    keepspaces=true,                 
    numbers=left,                    
    numbersep=5pt,                  
    showspaces=false,                
    showstringspaces=false,
    showtabs=false,                  
    tabsize=2
}
 
\lstset{style=mystyle}

\begin{document}
    \maketitle
    \tableofcontents
    \lstlistoflistings

    \section{Introduction}
    This is an assignment about guessing games and appropriate strategies. This
    is linked to searching algorithms, although in this case it's a sort of
    searching algorithm where you're searching for the item, rather than
    searching for the index of a known item.

    Both binary searches and linear
    searches can be adapted to that end, as both work by comparisons of the
    target and ``guesses''. These are analogous to asking the user questions - in
    the case of linear search, the question ``is your number equal to \ldots?'',
    and in the case of binary search, ``is your number greater than \ldots?''

    Therefore, essentially we need only to implement binary and linear search
    where the ``comparison function'' delegates to the user.  However,
    interpolation search, which has the complexity $O(\log\log n)$ (on uniform
    data sets, which we can assume a user-generated target is), requires
    knowledge of the \textit{value} of the target, so it can be interpolated on
    the current upper and lower bounds. This means we will only achieve the
    $O(\log n)$ complexity of binary search.

    Code for this assignment has been written in Pascal, targeted at and tested
    in fpc (version 3.0.0+dfsg-2), using mode \verb|{$MODE OBJFPC}|. This
    document was developed in \LaTeX.

    \section{Boilerplate}
    To create a program that performs such a search to guess a number, some
    utility functions will be needed, that aren't really particularly
    algorithmic. I'm implementing these in a Pascal unit, named
    \verb|PUser.pas|.

    \subsection{oracle}
    This is a function that can retrieve a boolean value from the user. It's
    named ``oracle'' as it acts as the comparison oracle the search functions
    must appeal to. The only parameter it takes is a message to display. The
    unix-inspired ruling it uses is that if an input starts with a ``y'', it is
    considered affirmative, and negative otherwise. The broad approach it uses
    is like this:

\begin{lstlisting}[caption=oracle pseudocode]
If the user's response is empty, the result is negative
If the user's response is not empty and begins with a lower or uppercase "y", the result is positive
Otherwise, the result is negative
\end{lstlisting}

    The catch for lower or uppercase is implemented by coercing the entire
    string to lowercase, using \verb|LowerCase|. It's implemented as:

\begin{lstlisting}[language=Pascal, caption=oracle implementation]
function oracle(msg: string): boolean;
var
    usr_input: string;
begin
    write(msg);
    readln(usr_input);
    if (length(usr_input) > 0) and (LowerCase(usr_input)[1] = 'y') then
        oracle := true
    else
        oracle := false;
end;
\end{lstlisting}

    NB: The lack of semicolon in line 8 is on purpose, as this is how an
    if-else statement works in Pascal (as it's considered one statement, the
    whole affair needs one semicolon).

    \subsection{read\_lohi}
    I wanted to provide some facility for setting the upper bound and lower
    bound of a binary search. As this function wants to kind of ``return'' two
    things, it has a nontrivial call signature, which is worth thinking about.
    One possible solution might be to define some kind of \verb|Bounds|
    container class that can hold both values.
    
    However, Pascal provides a way to output multiple values, and this is
    through an \verb|out| parameter. It could also be done using a \verb|var|
    parameter, although there's a slight difference. Both of these work, and in
    both cases the variable is passed by reference, but \verb|out| is slightly
    more specific - it's used when the input value of the variable is unneeded,
    which in this case is true, as I'm only concerned with using it as a channel
    for output. Once I've declared something as an \verb|out| parameter, I can
    assign to it and the caller can then use the new value (like the
    \verb|readln| function).

    I will be reading these from the command line arguments, as I'm developing
    it as a console application. As this is quite a simple application, I will
    just manually parse arguments with some if statements. The function will
    roughly do the following:

\begin{lstlisting}[caption=read\_lohi pseudocode]
Gather default values for the upper bound and lower bound
If a first command line argument is present
    Set the lower bound to this number
Otherwise, set it to the default
If a second command line argument is present
    Set the upper bound to this number
Otherwise, set it to the default
If either argument given was not an integer
    Set both to the defaults
\end{lstlisting}

    This is then implemented in Pascal using \verb|ParamStr| and
    \verb|ParamCount|. Interestingly, as it doesn't return anything in the
    conventional manner, this isn't a function but a procedure.

\begin{lstlisting}[language=Pascal, caption=read\_lohi implementation]
procedure read_lohi(lo_default, hi_default: integer;
                    out low_val, hi_val: integer);
begin
    try
        if ParamCount >= 1 then
            low_val := StrToInt(ParamStr(1))
        else
            low_val := lo_default;
        if ParamCount >= 2 then
            hi_val := StrToInt(ParamStr(2))
        else
            hi_val := hi_default;
    except
        on E: EConvertError do begin
            writeln('Conversion error occurred, reverting to defaults');
            low_val := lo_default;
            hi_val := hi_default;
        end;
    end;
end;
\end{lstlisting}

    \section{Linear Search}
    One approach to this is by a ``linear search''. This involved, basically, a
    kind of ``brute force'' approach - sequentially making guesses until one is
    correct.

    \section{Source}
    The full project in its directory structure, including this document, can
    be found at \url{https://github.com/elterminad0r/assignment_guessing}.

\end{document}
